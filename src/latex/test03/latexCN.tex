%导言区——进行全局设置
\documentclass{article} %有且仅有一个,可设置为article,book, report, letter

\usepackage{ctex}	%引入ctex宏包输出中文 编辑器中编码格式设置为UTF-8,编译器选择为XeLaTeX

%定义新命令:{^\circ}-->{\degree}
\newcommand{\degree}{^\circ}

\title{\heiti 杂谈勾股定理}
\author{\kaishu 李勇}
\date{\today}

%正文区(文稿区)
\begin{document}	%一个latex文件有且仅有一个document环境
	\maketitle	%输出标题信息,与letter类不兼容
	
	勾股定理可以用现代语言表述如下:
	直角三角形斜边的平方等于两腰的平方和。可以用符号语言表述为:设直角三角形$ABC$,其中$\angle C=90\degree$,则有:
	\begin{equation}	%带编号的行间公式
		AB^2=BC^2+AC^2 
	\end{equation}
\end{document}