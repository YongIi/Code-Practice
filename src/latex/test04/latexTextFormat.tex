%导言区——进行全局设置
\documentclass[10pt]{article} %[]是可选项10、11、12,设置normal size的字体大小

\usepackage{ctex}	%引入ctex宏包输出中文 编辑器中编码格式设置为UTF-8,编译器选择为XeLaTeX

%latex思想是格式与命令分离,不建议在文档中使用大量命令,一般定义一个新的命令执行相应的操作
\newcommand{\myfont}{\textit{\textbf{\textsf{Fancy Text}}}}

%正文区(文稿区)
\begin{document}
	% 1.1字体族设置(罗马字体、无衬线字体、打字机字体)
	\textrm{Roman Family} \textsf{Sans Serif Family} \texttt{Typewriter Family}
	
	% 1.2字体声明,声明后续的字体为罗马字体,{}可以限定作用范围
	\rmfamily Roman Family {\sffamily Sans Serif Family} {\ttfamily Typewriter Family}
	
	\rmfamily good good study
	
	{\ttfamily good good study}
	
	% 2字体系列设置(粗细、宽度)
	\textmd{Medium Series}	\textbf{Boldface Series}
	
	{\mdseries Medium Series}	{\bfseries Boldface Series}
	
	% 3.1字体形状(直立、斜体、伪斜体、小型大写)
	\textup{Upright Shape} \textit{Italic Shape}
	\textsl{Slanted Shape} \textsc{Small Caps Shape}
	
	{\upshape Upright Shape} {\itshape Italic Shape} {\slshape Slanted Shape} {\scshape {Small Caps Shape}
		
	% 3.2中文字体 \quad是空格
	{\songti 宋体} \quad {\heiti 黑体} \quad {\fangsong 仿宋} \quad {\kaishu 楷书}
	
	中文字体中的\textbf{粗体}与\textit{斜体}
	
	% 字体大小,是相对normal size而言的,其在\documentclass[10pt]{article}中指定
	{\tiny				hello}
	{\scriptsize		hello}
	{\footnotesize		hello}
	{\small				hello}
	{\normalsize		hello}
	{\large				hello}
	{\Large				hello}
	{\LARGE				hello}
	{\huge				hello}
	{\Huge				hello}
	
	% 中文字号设置命令,参数-0代表小初号
	\zihao{-0} 你好!
	
	\myfont
	
\end{document}