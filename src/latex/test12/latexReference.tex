%导言区——进行全局设置
\documentclass{article} %有且仅有一个,可设置为article,book, report, letter

\usepackage{ctex}	%引入ctex宏包输出中文 编辑器中编码格式设置为UTF-8,编译器选择为XeLaTeX
\bibliographystyle{plain} %调用数据库内的文献,有如下几种格式:plain unset alpha abbrb

\begin{document}
	% 一次管理,一次使用
	% 参考文献格式:
%	\begin{thebibliography}{编号样本}
%		\bibitem[记号]{引用标志}文献条目1
%		\bibitem[记号]{引用标志}文献条目2
%	\end{thebibliography}
%	其中文献条目包括:作者,题目,出版社,年代,页码等
%	引用时可以采用:\cite{引用标志1, 引用标志2}

	引用第一篇文献\cite{article1},引用第一本书\cite{book1}等等
	\begin{thebibliography}{99}
			\bibitem{article1}陈立辉, 苏伟.\emph{基于LaTex的Web数学公式提取方法研究}[J]. 计算机科学. 2014(06)
			\bibitem[记号]{引用标志}文献条目2
			\bibitem{book1}William H. Press, Saul A. Teukolsky. \emph{Numeriacal Recipes 3rd Edition: The Art of Scientific Computing} Cambridge University Press, New York, 2007.
	\end{thebibliography}

	% 强烈推荐使用第二种方法:一次管理,多次使用。
	采用一次管理,多次使用的方法。参考的文献放在参考文献库( RefData.bib文件)中,该文献被引用的时会自己列入文末的参考文献中。这种方式很方便,例如我引用参考文献\cite{sezen2021alternative}和参考文献\cite{pewowaruk2021solution}时,这两篇文献会自动出现在文末的参考文献中,不需要手动输入参考文献的作者,名称等等。参考文献的格式可以直接在Google scholar中复制BitTex的格式即可。
	\bibliography{RefData} % 不需要写RefData.bib的后缀.bib,RefData.bib文件的路径要与当前路径一致,若有多个数据库可以用逗号,隔开
	
	%若想在参考文献列表中排版未引用的文献,可以用\nocite{*}排版出所有(剩下的未引用的?)文献
	\nocite{*}


\end{document}