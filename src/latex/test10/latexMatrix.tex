%导言区——进行全局设置
\documentclass{article} %有且仅有一个,可设置为article,book, report, letter

\usepackage{ctex}	%引入ctex宏包输出中文 编辑器中编码格式设置为UTF-8,编译器选择为XeLaTeX
\usepackage{amsmath}	%引入该宏包才能使用matrix
\usepackage{mathdots}	%使用\iddots的反斜点

\begin{document}
	%在矩阵环境中,用&分隔列,用\\分隔行
	\[	%用中括号表示行间公式,其间不能随便加空行
	\begin{matrix}
		0 & 1 \\
		1 & 0
	\end{matrix}	\qquad
	\begin{pmatrix}		%矩阵两边加()
	0 & 1 \\
	1 & 0
	\end{pmatrix}	\qquad
	\begin{bmatrix}		%矩阵两边加[]
	0 & 1 \\
	1 & 0
	\end{bmatrix}	\qquad
	\begin{Bmatrix}		%矩阵两边加{}
	0 & 1 \\
	1 & 0
	\end{Bmatrix}	\qquad
	\begin{vmatrix}		%矩阵两边加||
	0 & 1 \\
	1 & 0
	\end{vmatrix}	\qquad
	\begin{Vmatrix}		%矩阵两边加|| ||
	0 & 1 \\
	1 & 0
	\end{Vmatrix}	\qquad
	\]
	
	% 矩阵里可以使用上下标
	\[
	A = \begin{pmatrix}
		a_{11}^2 & a_{12}^2 & a_{13}^2 \\
		0 & a_{22} & a_{23} \\
		0 & 0 & a_{33}
	\end{pmatrix}
	\]
	
	% 矩阵中常用的省略号: \dots \vdots \ddots
	\[
	A= \begin{bmatrix}
		a_{11} & \dots & a_{1n} \\
		\iddots & \ddots & \vdots \\  % \iddots仅用于展示反斜点
		0 & & a_{nn}
	\end{bmatrix}_{n \times n}
	\]
	
	% 分块矩阵(嵌套矩阵)
	\[
	\begin{pmatrix}
		\begin{matrix}
			1&0\\0&1
		\end{matrix} & \text{\Large 0} \\	%\text{}是在数学模式中临时切换到文本模式
	\text{\Large 0} & \begin{matrix}
		1&0\\0&-1\end{matrix}
	\end{pmatrix}
	\]
	
	% 三角矩阵
	\[
	\begin{pmatrix}
		a_{11} & a_{12} & \cdots & a_{1n} \\
		& a_{22} & \cdots & a_{2n} \\
		& & \ddots & \vdots \\
		& & & a_{nn}
	\end{pmatrix}
	\]
	或者
	%\multicolumn可以合并多列,2是合并的列数,c是居中。\raisebox调整高度
	\[
	\begin{pmatrix}
		a_{11} & a_{12} & \cdots & a_{1n} \\
		& a_{22} & \cdots & a_{2n} \\
		& & \ddots & \vdots \\
		\multicolumn{2}{c}{\raisebox{1.3ex}[0pt]{\Huge 0}}	
		& & a_{nn}
	\end{pmatrix}
	\]
	
	% 跨列的省略号:\hdotsfor{<列数>}
	\[
	\begin{pmatrix}
		1 & \frac 12 & \dots & \frac 1n \\
		\hdotsfor{4} \\
		m & \frac m2 & \dots & \frac mn
	\end{pmatrix}
	\]
	
	% 行内小矩阵 smallmatrix
	复数 $z = (x, y)$ 也可以用矩阵\begin{math}
		\left( 		%需要手动加上左括号
		\begin{smallmatrix}
			x & -y \\ y & x
		\end{smallmatrix}
		\right) 	%需要手动加上右括号
	\end{math}来表示.

	% 用array来表示矩阵(用法与表格tabular一致)
	\[
	\begin{array}{c|c}
		\frac 12 & 0 \\
		\hline
		0 & -\frac abc \\	%不用{}分组时,\frac仅能区分单个字母
	\end{array}
	\]
	
	% 用array可以构造复杂矩阵
	\[
	% @{<内容>} 添加任意内容,不占表项计数
	% 此处添加一个负值空白,表示向左移-5pt的距离
	\begin{array}{c@{\hspace{-5pt}}l}	%c是左边带括号的一列矩阵居中对齐,中间是个-5磅的空白,右边是居左对齐
		% 第1行,第1列
		\left(
		\begin{array}{ccc|ccc}
			a & \cdots & a & b & \cdots & b \\
			&  \ddots & \vdots & \vdots & \iddots & \\
			& & a & b \\ \hline
			& &   & c & \cdots & c \\
			& &   & \vdots & & \vdots \\
			\multicolumn{3}{c|}{\raisebox{2ex}[0pt]{\Huge 0}} & c & \cdots & c
		\end{array}
		\right)
		&
		% 第1行第2列
		\begin{array}{l}
			%$\left.仅表示与$\right\}匹配,什么都不输出
			\left.\rule{0mm}{7mm}\right\}p \\
			\\
			\left.\rule{0mm}{7mm}\right\}q
		\end{array}
	\\[-5pt]
	% 第2行第1列
	\begin{array}{cc}
		\underbrace{\rule{17mm}{0mm}}_m &
		\underbrace{\rule{17mm}{0mm}}_m
	\end{array}
	% 第2行第2列
	\end{array}
	\]
	
\end{document}