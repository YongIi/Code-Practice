%导言区——进行全局设置
\documentclass{article} %有且仅有一个,可设置为article,book, report, letter

\usepackage{ctex}	%引入ctex宏包输出中文 编辑器中编码格式设置为UTF-8,编译器选择为XeLaTeX
\usepackage{amsmath}	%用于使用equation*环境
\usepackage{amssymb}	%用于花体字符

% 正文区
\begin{document}
	% 带编号
	\begin{gather}
		\sqrt{x^2+y^2} \\
		\frac{1}{1+\frac{1}{x}}
	\end{gather}

	% 不带编号
	\begin{gather*}
		\sqrt{x^2+y^2} \\
		\frac{1}{1+\frac{1}{x}}
	\end{gather*}

	% 在\\前面使用\notag阻止编号
	\begin{gather}
		3^2+4^2=5^2 \notag \\
		\frac{1}{1+\frac{1}{x}} \notag \\
		\sqrt{x^2+y^2}
	\end{gather}

	% align和align*,用&进行对齐 (例如=对齐,或者公式的起始位置对齐——&放在起始位置)
	% 带编号
	\begin{align}
		3^2+4^2 &= 5^2 \\
		a &= \sqrt{\frac{x}{x^2+x+1}}
	\end{align}
	% 不带编号
	\begin{align*}
		3^2+4^2 &= 5^2 \\
		a &= \sqrt{\frac{x}{x^2+x+1}}
	\end{align*}

	% split实现一个公式的多行排版,在\begin{equation}中
	\begin{equation}
		\begin{split}
			\cos 2x &= \cos^2 x - \sin^2 x \\
			&= 2\cos^2 x - 1
		\end{split}
	\end{equation}

	% 分段函数的排版——cases环境
	% 每行公式中用&分隔为两部分,通常表示值和后面的条件
	\begin{equation}
		D(x)= \begin{cases}
			1, & \text{如果 } x \in \mathbb{Q}; \\ % mathbb用于花体字符
			0, & \text{如果 } x \in \mathbb{R}\setminus\mathbb{Q}.
		\end{cases}
	\end{equation}
	
\end{document}










